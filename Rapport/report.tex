% Class and style of the document.
\documentclass[12pt,a4paper]{article}

% Package and configuration.

% Language and encoding packages.
\usepackage[french]{babel}
\usepackage[utf8]{inputenc}
\usepackage[T1]{fontenc}

% Table package.
\usepackage{csvsimple}
\usepackage{booktabs}
\usepackage{changepage}

% Mathematics package.
\usepackage{amsmath}
\usepackage{amstext}

% Figures package.
\usepackage{graphicx}
\usepackage{wrapfig}

% Charts package.
\usepackage{pgfplots}
\pgfplotsset{compat=1.16}
\usepgfplotslibrary{units}

% Symbols and numbers.
\usepackage{siunitx}
\usepackage{numprint}

% Formatter package.
\usepackage{pdflscape}
\usepackage{vmargin}
%% Page break between section.
\usepackage{titlesec}
\newcommand{\sectionbreak}{\clearpage}
%% Use clickable URLs.
\usepackage{url}

% Hypertext links.
\usepackage[hidelinks]{hyperref}

% Footer.
\usepackage{lastpage}
\usepackage{fancyhdr}
\pagestyle{fancy}
% Clear headers and footers.
\fancyhead{}
% \renewcommand{\headrulewidth}{0pt}
% \renewcommand{\footrulewidth}{0pt}
% Define footers.
\fancyfoot[RE,LO]{Pierre AYOUB, Océane FLAMANT, Dorsaf FRIGUI}
\fancyfoot[RE,CO]{}
\fancyfoot[RE,RO]{\text{\thepage} sur \pageref{LastPage}}

% Paragraph.
\setlength{\parskip}{1ex}

% Quote.
\usepackage[style=french,french=guillemets]{csquotes}

\begin{document}

\title{Gestion des stages dans une école d'ingénieurs}
\author{Pierre AYOUB -- Océane FLAMANT -- Dorsaf FRIGUI}

\maketitle

\begin{figure}[b]
    \centering
    \includegraphics[scale=0.3]{pictures/isty.jpg}
\end{figure}

\tableofcontents

\section{Introduction}

Ce projet avait pour objectif la modélisation du processus de gestion des stages
d'une école d'ingénieur en appliquant la méthode CLHYPS. Cette dernière, dont
l'acronyme est inconnu de ses utilisateurs, est un cadre méthodologique alliant
modèle objet et approche systémique en proposant des méthodes et des algorithmes
permettant de modéliser un système.

La modélisation de ce système s'est déroulée en plusieurs étapes :
\begin{itemize}
        \item Analyse conceptuelle : cette première étape consiste en un
            dialogue avec le client qui nous expose ses différents besoins. De
            cette description du système, il nous incombe de définir les points
            pertinents à traiter, ainsi que les besoins sensibles et les axes de
            travail.
        \item Analyse des exigences fonctionnelles : c'est durant cette deuxième
            étape que le travail formel commence. En partant de notre
            analyse et compréhension des besoins du client grâce à notre étape
            précédente, nous pouvons maintenant dresser un tableau d'exigences
            représentant notre système.
        \item Modèle objet : depuis notre tableau des exigences fonctionnelles,
            nous allons maintenant pouvoir en \enquote{dériver} une
            représentation du système selon le modèle objet. Cette
            représentation nous permet de voir les différentes classes de
            notre système, ainsi que les informations qu'elles portent.
        \item Modèle d'activité : encore depuis notre tableau des exigences
            fonctionnelles, nous allons pouvoir \enquote{dériver} une
            représentation d'activité du système. Ce modèle nous permet de
            justifier les flux qui régissent la vie de ce dernier, en mettant en
            évidence les différentes interactions entre les agents du contexte.
\end{itemize}

\section{Modèles}

Pour des soucis de mise en page, nous utiliserons des abréviations pour désigner
les différents rôles des agents dans nos modèles. En voici la liste :
\begin{enumerate}
    \item À l'école :
        \begin{description}
            \item[RDS] {Responsable des stages}
            \item[DE]  {Direction de l'école}
            \item[CD]  {Chef du département de la spécialité}
            \item[RS]  {Reponsable de la scolarité}
            \item[CS]  {Comission de stage}
            \item[AM]  {Anciens membres de la comission de stage}
            \item[ENS] {Enseignants de l'école}
            \item[ELV] {Élèves}
        \end{description}
    \item En entreprise :
        \begin{description}
            \item[ST]  {Stagiaires}
            \item[TP]  {Tuteur pédagogique (est en fait un ENS de l'année en cours)}
            \item[EE]  {Encadrant en entreprise}
            \item[SCE] {Signataire de la convention dans l'entreprise}
        \end{description}
\end{enumerate}

\subsection{Modèle d'exigences fonctionnelles}

Ci-dessous se présente le modèle d'exigences fonctionnelles, séparé en deux
\enquote{packages}, eux-même séparés en plusieurs tableaux pour des raisons de
lisibilité. Chaque package correspond à une période de l'année et un cadre
totalement différent : le premier se passe à l'école, le second se passe en
entreprise. 

\subsubsection{Package A - Organisation de la campagne de stage}

Le premier package correspond à l'ensemble des sollications permettant
d'organiser la campagne de stage, que ce soit du côté de l'école, de
l'entreprise ou bien de l'élève.

Nous noterons que la sollicitation intitulée \enquote{ObtenirConvStage} est une
\enquote{méta-sollicitation} représentant l'ensemble des procédures et
sollicitations établies en TP.

\newpage \setpapersize[landscape]{A4}
\begin{adjustwidth}{-2cm}{-2cm}
    \begin{tabular}{|l|c|c|p{5cm}|p{9cm}|} \hline
        \bfseries Sollicitation & \bfseries Commanditaire & \bfseries Bénéficiaire & \bfseries Ressource (classe objet) & \bfseries Finalité
        \csvreader[separator=pipe, head to column names]{./tables/pack-a.csv}{}{\\ \hline \sol & \com & \ben & \res & \fin}
        \\ \hline
    \end{tabular}
\end{adjustwidth}
\newpage \setpapersize{A4}

\newpage \setpapersize[landscape]{A4}
\begin{adjustwidth}{-2cm}{-2cm}
    \begin{tabular}{|l|p{3cm}|p{3cm}|p{7cm}|p{7cm}|} \hline
        \bfseries Sollicitation & \bfseries Destinaires (Primaires / Secondaires) & \bfseries Condition d'émission & \bfseries Condition de recevabilité & \bfseries Type d'effet
        \csvreader[separator=pipe, head to column names]{./tables/pack-a.csv}{}{\\ \hline \sol & \des & \condemi & \condrec & \eff}
        \\ \hline
    \end{tabular}
\end{adjustwidth}
\newpage \setpapersize{A4}

\newpage \setpapersize[landscape]{A4}
\begin{tabular}{|l|l|c|c|} \hline
    \bfseries Sollicitation & \bfseries Procédure & \bfseries Point d'entrée & \bfseries Rang
    \csvreader[separator=pipe, head to column names]{./tables/pack-a.csv}{}{\\ \hline \sol & \proc & \ptentr & \rang}
    \\ \hline
\end{tabular}
\newpage \setpapersize{A4}

\subsubsection{Package B - Phase d'exécution du stage}

Le deuxième package correspond à l'ensemble des sollicitations se déroulant
durant la période de stage en entreprise. Tout ceci se passe donc après le
travail que nous avons effectué en TP.

\newpage \setpapersize[landscape]{A4}
\begin{adjustwidth}{-2.3cm}{-2.3cm}
    \begin{tabular}{|p{5.5cm}|c|c|p{6cm}|p{8cm}|} \hline
        \bfseries Sollicitation & \bfseries Commanditaire & \bfseries Bénéficiaire & \bfseries Ressource (classe objet) & \bfseries Finalité
        \csvreader[separator=pipe, head to column names]{./tables/pack-b.csv}{}{\\ \hline \sol & \com & \ben & \res & \fin}
        \\ \hline
    \end{tabular}
\end{adjustwidth}
\newpage \setpapersize{A4}

\newpage \setpapersize[landscape]{A4}
\begin{adjustwidth}{-2.5cm}{-2.5cm}
    \begin{tabular}{|p{5.5cm}|p{3cm}|p{3cm}|p{7cm}|p{7cm}|} \hline
        \bfseries Sollicitation & \bfseries Destinaires (Primaires / Secondaires) & \bfseries Condition d'émission & \bfseries Condition de recevabilité & \bfseries Type d'effet
        \csvreader[separator=pipe, head to column names]{./tables/pack-b.csv}{}{\\ \hline \sol & \des & \condemi & \condrec & \eff}
        \\ \hline
    \end{tabular}
\end{adjustwidth}
\newpage \setpapersize{A4}

\newpage \setpapersize[landscape]{A4}
\begin{tabular}{|l|l|c|c|} \hline
    \bfseries Sollicitation & \bfseries Procédure & \bfseries Point d'entrée & \bfseries Rang
    \csvreader[separator=pipe, head to column names]{./tables/pack-b.csv}{}{\\ \hline \sol & \proc & \ptentr & \rang}
    \\ \hline
\end{tabular}
\newpage \setpapersize{A4}

\subsection{Modèle d'activité}

\subsection{Modèle objet}

\section{Limites}

\section{Conclusion}

\end{document}
