% Class and style of the document.
\documentclass[12pt,a4paper]{article}

% Package and configuration.

% Language and encoding packages.
\usepackage[french]{babel}
\usepackage[utf8]{inputenc}
\usepackage[T1]{fontenc}

% Table package.
\usepackage{csvsimple}
\usepackage{booktabs}
\usepackage{array}
\usepackage[table]{xcolor}

% Mathematics package.
\usepackage{amsmath}
\usepackage{amstext}

% Figures package.
\usepackage{graphicx}
\usepackage{wrapfig}
\usepackage{svg}

% Charts package.
\usepackage{pgfplots}
\pgfplotsset{compat=1.16}
\usepgfplotslibrary{units}

% Symbols and numbers.
\usepackage{siunitx}
\usepackage{numprint}

% Formatter package.
\usepackage{geometry}
\usepackage{pdflscape}
\usepackage{vmargin}
%% Page break between section.
\usepackage{titlesec}
\newcommand{\sectionbreak}{\clearpage}
%% Use clickable URLs.
\usepackage{url}

% Hypertext links.
\usepackage[hidelinks]{hyperref}

% Footer.
\usepackage{lastpage}
\usepackage{fancyhdr}
\pagestyle{fancy}
% Clear headers and footers.
\fancyhead{}
\renewcommand{\headrulewidth}{0pt}
\renewcommand{\footrulewidth}{0pt}
% Define footers.
\fancyfoot[RE,LO]{Pierre AYOUB, Océane FLAMANT, Dorsaf FRIGUI}
\fancyfoot[RE,CO]{}
\fancyfoot[RE,RO]{\text{\thepage} sur \pageref{LastPage}}

% Paragraph.
\setlength{\parskip}{1ex}

% Quote.
\usepackage[style=french,french=guillemets]{csquotes}

\begin{document}

\title{Gestion des stages dans une école d'ingénieurs}
\author{Pierre AYOUB -- Océane FLAMANT -- Dorsaf FRIGUI}

\maketitle

\begin{figure}[b]
    \centering
    \includegraphics[scale=0.3]{figures/isty.jpg}
\end{figure}

\tableofcontents

\section{Introduction}

Ce projet avait pour objectif la modélisation du processus de gestion des stages
d'une école d'ingénieur en appliquant la méthode CLHYPS. Cette dernière, dont
l'acronyme est inconnu de ses utilisateurs, est un cadre méthodologique alliant
modèle objet et approche systémique en proposant des méthodes et des algorithmes
permettant de modéliser un système.

La modélisation de ce système s'est déroulée en plusieurs étapes :
\begin{itemize}
        \item Analyse conceptuelle : cette première étape consiste en un
            dialogue avec le client qui nous expose ses différents besoins. De
            cette description du système, il nous incombe de définir les points
            pertinents à traiter, ainsi que les besoins sensibles et les axes de
            travail.
        \item Analyse des exigences fonctionnelles : c'est durant cette deuxième
            étape que le travail formel commence. En partant de notre
            analyse et compréhension des besoins du client grâce à notre étape
            précédente, nous pouvons maintenant dresser un tableau d'exigences
            représentant notre système.
        \item Modèle d'activité : encore depuis notre tableau des exigences
            fonctionnelles, nous allons pouvoir \enquote{dériver} une
            représentation d'activité du système. Ce modèle nous permet de
            justifier les flux qui régissent la vie de ce dernier, en mettant en
            évidence les différentes intéractions entre les agents du contexte.
        \item Modèle objet : depuis notre tableau des exigences fonctionnelles,
            nous allons maintenant pouvoir en \enquote{dériver} une
            représentation du système selon le modèle objet. Cette
            représentation nous permet de voir les différentes classes de
            notre système, ainsi que les informations qu'elles portent.
\end{itemize}

\section{Modèles}

Pour des soucis de mise en page, nous utiliserons des abréviations pour désigner
les différents rôles des agents dans nos modèles. En voici la liste :
\begin{enumerate}
    \item À l'école :
        \begin{description}
            \item[RDS] {Responsable des stages}
            \item[DE]  {Direction de l'école}
            \item[CD]  {Chef du département de la spécialité}
            \item[RS]  {Reponsable de la scolarité}
            \item[CS]  {Comission de stage}
            \item[AM]  {Anciens membres de la comission de stage}
            \item[ENS] {Enseignants de l'école}
            \item[ELV] {Élèves}
        \end{description}
    \item En entreprise :
        \begin{description}
            \item[ST]  {Stagiaires}
            \item[TP]  {Tuteur pédagogique (est en fait un ENS de l'année en cours)}
            \item[EE]  {Encadrant en entreprise}
            \item[SCE] {Signataire de la convention dans l'entreprise}
        \end{description}
\end{enumerate}

\subsection{Modèle d'exigences fonctionnelles}

Ci-dessous se présente le modèle d'exigences fonctionnelles, séparé en deux
\enquote{packages}, eux-même séparés en plusieurs tableaux pour des raisons de
lisibilité. Chaque package correspond à une période de l'année et un cadre
totalement différent : le premier se passe à l'école, le second se passe en
entreprise. 

\subsubsection{Package A - Organisation de la campagne de stage}

Le premier package correspond à l'ensemble des sollications permettant
d'organiser la campagne de stage, que ce soit du côté de l'école, de
l'entreprise ou bien de l'élève.

Nous noterons que la sollicitation intitulée \enquote{ObtenirConvStage} est une
\enquote{méta-sollicitation} représentant l'ensemble des procédures et
sollicitations établies en TP.

\newpage \setpapersize[landscape]{A4}
\newgeometry{left=3.5cm, top=6.2cm}
{
    \rowcolors{2}{gray!40}{gray!10}
    \setlength{\arrayrulewidth}{2pt}
    \begin{tabular}{|l|c|c|m{5cm}|m{9cm}|} \hline
        \rowcolor{gray!60} \bfseries Sollicitation & \bfseries Commanditaire & \bfseries Bénéficiaire & \bfseries Ressource (classe objet) & \bfseries Finalité
        \csvreader[separator=pipe, head to column names]{./tables/modele-exigence-pack-a.csv}{}{\\ \sol & \com & \ben & \res & \fin}
        \\ \hline
    \end{tabular}
}
\newpage \setpapersize{A4}
\restoregeometry

\newpage \setpapersize[landscape]{A4}
\newgeometry{left=3.5cm, top=5.8cm}
{
    \rowcolors{2}{gray!40}{gray!10}
    \setlength{\arrayrulewidth}{2pt}
    \begin{tabular}{|l|m{3cm}|m{3cm}|m{7cm}|m{7cm}|} \hline
        \rowcolor{gray!60} \bfseries Sollicitation & \bfseries Destinaires (Primaires / Secondaires) & \bfseries Condition d'émission & \bfseries Condition de recevabilité & \bfseries Type d'effet
        \csvreader[separator=pipe, head to column names]{./tables/modele-exigence-pack-a.csv}{}{\\ \sol & \des & \condemi & \condrec & \eff}
        \\ \hline
    \end{tabular}
}
\newpage \setpapersize{A4}
\restoregeometry

\newpage \setpapersize[landscape]{A4}
\newgeometry{left=6.4cm, top=8.7cm}
{
    \rowcolors{2}{gray!40}{gray!10}
    \setlength{\arrayrulewidth}{2pt}
    \begin{tabular}{|l|l|c|c|} \hline
        \rowcolor{gray!60} \bfseries Sollicitation & \bfseries Procédure & \bfseries Point d'entrée & \bfseries Rang
        \csvreader[separator=pipe, head to column names]{./tables/modele-exigence-pack-a.csv}{}{\\ \sol & \proc & \ptentr & \rang}
        \\ \hline
    \end{tabular}
}
\newpage \setpapersize{A4}
\restoregeometry

\subsubsection{Package B - Phase d'exécution du stage}

Le deuxième package correspond à l'ensemble des sollicitations se déroulant
durant la période de stage en entreprise. Tout ceci se passe donc après le
travail que nous avons effectué en TP.

\newpage \setpapersize[landscape]{A4}
\newgeometry{left=3cm, top=4.5cm}
{
    \rowcolors{2}{gray!40}{gray!10}
    \setlength{\arrayrulewidth}{2pt}
    \begin{tabular}{|m{5.5cm}|c|c|m{6cm}|m{8cm}|} \hline
        \rowcolor{gray!60} \bfseries Sollicitation & \bfseries Commanditaire & \bfseries Bénéficiaire & \bfseries Ressource (classe objet) & \bfseries Finalité
        \csvreader[separator=pipe, head to column names]{./tables/modele-exigence-pack-b.csv}{}{\\ \sol & \com & \ben & \res & \fin}
        \\ \hline
    \end{tabular}
}
\newpage \setpapersize{A4}
\restoregeometry

\newpage \setpapersize[landscape]{A4}
\newgeometry{left=2.8cm, top=5.7cm}
{
    \rowcolors{2}{gray!40}{gray!10}
    \setlength{\arrayrulewidth}{2pt}
    \begin{tabular}{|m{5.5cm}|m{3cm}|m{3cm}|m{7cm}|m{7cm}|} \hline
        \rowcolor{gray!60} \bfseries Sollicitation & \bfseries Destinaires (Primaires / Secondaires) & \bfseries Condition d'émission & \bfseries Condition de recevabilité & \bfseries Type d'effet
        \csvreader[separator=pipe, head to column names]{./tables/modele-exigence-pack-b.csv}{}{\\ \sol & \des & \condemi & \condrec & \eff}
        \\ \hline
    \end{tabular}
}
\newpage \setpapersize{A4}
\restoregeometry

\newpage \setpapersize[landscape]{A4}
\newgeometry{left=6.4cm, top=7.4cm}
{
    \rowcolors{2}{gray!40}{gray!10}
    \setlength{\arrayrulewidth}{2pt}
    \begin{tabular}{|l|l|c|c|} \hline
        \rowcolor{gray!60} \bfseries Sollicitation & \bfseries Procédure & \bfseries Point d'entrée & \bfseries Rang
        \csvreader[separator=pipe, head to column names]{./tables/modele-exigence-pack-b.csv}{}{\\ \sol & \proc & \ptentr & \rang}
        \\ \hline
    \end{tabular}
}
\newpage \setpapersize{A4}
\restoregeometry

\subsection{Modèle d'activité}

Nous allons maintenant vous présenter le modèle d'activité de notre système,
toujours séparé en deux packages, avec chaque package séparé en plusieurs
parties. Ce modèle nous permet de mettre en évidence les interactions des
différents agents du système, ainsi que les relations entre ces derniers grâce
au passage de ressources.

\subsubsection{Package A - Organisation de la campagne de stage}

Le package A est organisé en deux parties, pour des raisons de visiblité. La
première partie décrit les nombreuses étapes de la procédure de création de la
commission de stage : la demande aux anciens membres, la demande aux
enseignants, enfin la formation de la commission. La deuxième partie modélise le
reste de l'établissement de la campagne de stage : la fixation du nombre
d'élèves maximum par enseignant, le lancement officiel de la campagne, ainsi que
la clôture de la campagne avec l'affectation des enseignants aux élèves en tant
que tuteur pédagogique.

{
    \newpage \setpapersize[landscape]{A4} \newgeometry{left=3cm, top=3.7cm}
    \includesvg[scale=0.94]{./figures/modele-activite-pack-a-1.svg}
    \restoregeometry \setpapersize{A4} \newpage
}

{
    \newpage \setpapersize[landscape]{A4} \newgeometry{left=3cm, top=3.7cm}
    \includesvg[scale=0.94]{./figures/modele-activite-pack-a-2.svg}
    \restoregeometry \setpapersize{A4} \newpage
}

\subsubsection{Package B - Phase d'exécution du stage}

Le package B est, lui, composé en 3 parties. La première partie décrit les
sollicitations concernant les différents rapports destinés à circuler : le
procès verbal d'intégration, le rapport d'installation, le rapport de mi-stage et
finalement le rapport final. Chacun de ces rapports, hormis le rapport
d'installation, peut faire l'objet d'un rappel. La deuxième partie représente
les traitements affectants la visite : l'exécution de la visite, l'envoi du
rapport de la visite, ainsi que la demande d'annulation de la visite. La
troisième partie correspond à la demande d'interruption du stage avec toutes les
procédures qui en découlent.

{
    \newpage \setpapersize[landscape]{A4} \newgeometry{left=4cm, top=3.9cm}
    \includesvg[scale=0.94]{./figures/modele-activite-pack-b-1.svg}
    \restoregeometry \setpapersize{A4} \newpage
}

{
    \newpage \setpapersize[landscape]{A4} \newgeometry{left=4.5cm, top=4.5cm}
    \includesvg{./figures/modele-activite-pack-b-2.svg}
    \restoregeometry \setpapersize{A4} \newpage
}

{
    \newpage \setpapersize[landscape]{A4} \newgeometry{left=3cm, top=3.9cm}
    \includesvg[scale=0.94]{./figures/modele-activite-pack-b-3.svg}
    \restoregeometry \setpapersize{A4} \newpage
}

\subsection{Modèle objet}

En dernier lieu, nous pouvons dès à présent analyser le modèle objet du système,
séparé en deux packages et plusieurs parties de la même manière que les deux
modèles précédents. Ce modèle décrit le système sous un regard orienté-objet
et utilise le formalisme d'UML pour mettre en place un diagramme de classe. Nous
pouvons donc observer chaque agent et ressource se matérialiser sous la forme de
classe, possédant différentes méthodes afin de répondre aux sollicitations.

\subsubsection{Package A - Organisation de la campagne de stage}

Nous avons 3 parties : la première représente la formation de la commission de
stage, la deuxième partie représentant quant à elle la fixation du nombre
d'élèves maximum par tuteur pédagogique, le lancement officiel de la campagne de
stage, ainsi que la finalisation de cette derinère.

{
    \tiny
    \newpage \setpapersize[landscape]{A4} \newgeometry{left=7cm, top=4.5cm}
    \includesvg[scale=0.35]{./figures/modele-objet-pack-a-1.svg}
    \restoregeometry \setpapersize{A4} \newpage
}

{
    \newpage \setpapersize[landscape]{A4} \newgeometry{left=3.3cm, top=4cm}
    \includesvg[scale=0.70]{./figures/modele-objet-pack-a-2.svg}
    \restoregeometry \setpapersize{A4} \newpage
}

\subsubsection{Package B - Phase d'exécution du stage}

Le package B est lui aussi séparé en 3 parties : la première modélisant l'envoi
des différents rapports lors du stage, la deuxième les interactions possibles
concernant les visites, enfin la troisème représentant l'interruption d'un stage
en cours.

{
    \newpage \setpapersize[landscape]{A4} \newgeometry{left=5cm, top=5cm}
    \includesvg[scale=0.70]{./figures/modele-objet-pack-b-1.svg}
    \restoregeometry \setpapersize{A4} \newpage
}

{
    \newpage \setpapersize[landscape]{A4} \newgeometry{left=2cm, top=4cm}
    \includesvg[scale=0.63]{./figures/modele-objet-pack-b-2.svg}
    \restoregeometry \setpapersize{A4} \newpage
}

{
    \newpage \setpapersize[landscape]{A4} \newgeometry{left=2.2cm, top=5.5cm}
    \includesvg[scale=0.63]{./figures/modele-objet-pack-b-3.svg}
    \restoregeometry \setpapersize{A4} \newpage
}

\section{Limites}

Au cours de la modélisation, nous nous sommes confrontés à certains types de
problèmes :
\begin{enumerate}
    \item Mise en évidence du parallélisme : ces représentations nous permettent
        de décrire un système avec des évènements se déroulant en parallèle (ce
        que nous avons fait), cependant nous estimons que le parallélisme
        n'apparait pas assez clairement dans nos modèles.
    \item La survenance d'une sollicitation qui peut en egendrer une autre : par
        exemple, lors du rappel d'envoyer un rapport, le rappel peut ou non se
        produire, mais l'envoi doit toujours être fait. La modélisation de ce
        type de sollicitation nous a posé quelques problèmes concernant les
        rangs et les points d'entrée.
    \item Sollicitation avec plusieurs ressources : dans le cadre d'une
        sollication ayant une ressource obligatoire et une ressource
        facultative, nous ne savions pas vraiment comment modéliser ce
        phénomène.
\end{enumerate}

\section{Conclusion}

Ce projet nous aura permis de nous confronter à une modélisation de plus grande
ampleur que ce que nous avions fait en travaux dirigés. Concevoir un système de
A à Z avec \enquote{autant} d'agents et de sollicitations se révèle beaucoup
plus fastidieux et compliqué que la petite portion que nous avons fait en
travaux dirigés. Cependant, cela nous aura été bénéfique et nous a conforté dans
notre utilisation de la méthode CLHYPS. Tout en gardant à l'esprit le fait qu'il
nous reste sans doute beaucoup de choses à apprendre concernant ce cadre
méthodologique.

\end{document}
